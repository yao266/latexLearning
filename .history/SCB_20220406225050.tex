\documentclass[UTF8]{article}
\usepackage{CTEX}
\usepackage{amsmath}
\usepackage{subcaption}
\usepackage{graphicx}
\usepackage{siunitx}
\usepackage{booktabs}
\begin{document}
    \section{研究背景与问题重述}
    \subsection{研究背景}
    随着经济的快速发展,城市机动车保有量持续上升,居民出行机不可挡。随着交通运输水平的提高,对交通运输的总需求不断增加。如今,主要城市道路日益拥挤被认为是世界上普遍的现象一个社会问题,
    它的出现使得城市的可持续发展轨迹与人民的日常生活和工作秩序发生了严重的质的变化。
    道路基础设施增长速度远不如机动车的增长速度快需求与交通供给不平衡,导致交通频繁拥堵,给城市交通形势带来了新的挑战[2]。为了缓解交通拥堵造成的经济负增长和人们生活工作秩序的混乱,
    对交通状态进行评价是非常必要的。对于城市交通管理者来说,如何快速准确地发现交通拥堵,并根据拥堵程度采取有效的对策,对于缓解交通拥堵有着重要的作用。
    交通参与者可以根据交通状态信息选择更平坦的道路线路出行可以缓解路段拥挤的压力,达到减少交通拥堵的程度。\\
    道路交叉口是车辆和行人聚集、转弯和疏散的场所,是交通的咽喉。因此,正确设计交叉口道路,合理组织、管理交叉口交通,是提高道路通行能力和保证交通安全的重要手段。
    从交通拥堵评价的相关研究来看,现有的评价方法主要集中在道路和路网的评价上,相关的评价指标主要包括交叉路口饱和度、平均停车延误、交叉口速度比、交通密度等状态指标[3]。
    由于数据采集手段的限制,无法准确采集评价模型中的一些参数。虽然一些新的探测设备已经投入市场,但由于建设成本和探测效果的影响,国内城市还没有大规模安装。\\
    \subsection{问题重述}
    以上述背景为契机,题目提供了多功能电警数据(流量和车尾时距)。为了建立一个路口交通状态的评价模型,在题目已给的数据上进行数学建模,且需要回答以下两个问题:\\
    (1)基于所提供的数据,和现有的评价指标建立一个能够评价路口交通状态的数学模型。\\
    (2)基于所提供的数据和第一问的模型,建立数学模型用于路口信号灯配时方案的调优工作。
    \section{模型假设与符号说明}
    \subsection{模型假设}
    (1)假设汽车在经过交叉路口后分别在四条道路的区间速度是均匀的;\\
    (2)假设汽车在观山东路和长岭路交叉路口通过红绿灯的速度为最大限速30码;\\
    \subsection{符号说明}
    
    \begin{table}[h] 
        \centering
        \caption{符号说明}
         
         \label{table_time}
         \resizebox{\textwidth}{!}{
         \tiny 6pt 6pt 6pt
         \begin{tabular}{lllllll}
         
        \toprule  
         
          指标符号 & 符号说明  & 单位 \\
               
         
        \midrule  
         
          $\mathcal{Q}$  & 某一方向的车流驶入量 & ——  \\ 
         
          $\nu$  & 平均车尾时距 &   s  \\
          \bottomrule 
         
        \end{tabular}
         }
        \end{table}
    \subsection{测试}
    
\end{document}